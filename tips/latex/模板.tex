\documentclass[12pt,a4paper]{article}

%開啟設定字體
%\usepackage{fontspect}

\usepackage{xcolor}
\usepackage{listings}

%使用xeCJK package
\usepackage{xeCJK}

%用來打logo的package
\usepackage{metalogo}

%table of content
\usepackage{outline}

\usepackage{fancyhdr}
\pagestyle{fancy}
\lhead{中文\XeLaTeX 教學}
\rhead{\thepage}
%英文字型
\setmainfont{Times New Roman}

%設定中文字型
\setCJKmainfont{Droid Sans Fallback}

%---------------------------------------------------------%
%中文自動換行
\XeTeXlinebreaklocale "zh"

%文字的彈性間距
\XeTeXlinebreakskip = 0pt plus 1pt

%設定段落之間的距離
\setlength{\parskip}{0.3cm}

%設定行距
\linespread{1.5}\selectfont
%---------------------------------------------------------%



\title{中文 \XeLaTeX 教學}

\author{GGJason 吳軒竹}

\date{\today}

\lstset
{
    language=[LaTeX]TeX,
    breaklines=true,
    basicstyle=\tt\scriptsize,
    keywordstyle=\color{blue},
    identifierstyle=\color{magenta},
}


\begin{document}
	\maketitle
	\newpage
	\begin{abstract}
		這份教學的主要目的是希望提供中文 \LaTeX 及 \XeLaTeX 使用者更方便使用的指導,並且在最後附上快速查詢表,提供已經熟悉的使用者能夠快速的查到想要使用的功能。主要參考的是 TUG\footnote{\TeX  User Group}的資料以及我個人在各個部落格裡面翻出來的資訊,希望讓 \LaTeX 的門檻降低,提昇使用的人口。
	\end{abstract}
	\newpage
	\tableofcontents
	\newpage
	\part{ \XeLaTeX 基本使用教學}

		\section{ \XeLaTeX 簡介}
			\subsection{\TeX 與 \LaTeX}
				\TeX 是Donald Ervin Knuth教授所設計出來的排版軟體\footnote{中文維基百科} ,
			\subsection{\XeLaTeX}
		\section{ \XeLaTeX 環境設定}
			我在這邊所使用的是 Ubuntu 16.04 LTS,並且純粹以TeXworks(http://www.tug.org/texworks/)進行環境整合與編譯。後續有機會的話我會逐步補充Windows跟Mac OS的使用方法,不過除了安裝之外其實大多部份視差不多的。	
			Ubuntu 可以藉由 Ubuntu 軟體中心進行安裝,僅需在裡面搜尋,便會自動下載及安裝。
			\subsection{中文環境設置}
			$\backslash$usepackage $\{$xeCJK$\}$
		
		\section{基礎操作}
			\subsection{ Hello World }
			\begin{lstlisting}
\documentclass{article}
\begin{doucument}
Hello World!
\end{document}
			\end{lstlisting}
			\subsection{設定作者、日期、標題頁面}
			\begin{lstlisting} 
\title{標題}
\author{作者名稱}
\date{\today}
\maketitle
			\end{lstlisting}
			\subsection{段落}
				\begin{lstlisting}
\section{大段落}
\subsection{中段落}
\subsubsection{小段落}
				\end{lstlisting}
					\subsubsection{小段落}
				沒有更小的段落了
			\subsection{項目}
				\begin{itemize}
					\item 項目一 
					\item 項目二
				\end{itemize}
				\begin{enumerate}
					\item 項目一 
					\item 項目二
				\end{enumerate}
				\begin{description}
					\item [Ant] 螞蟻
					\item [Elephant]大象
				\end{description}
		\section{數學算式}
\end{document}
